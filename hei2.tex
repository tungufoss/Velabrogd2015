\documentclass[]{article}
\usepackage[T1]{fontenc}
\usepackage[icelandic]{babel}
\usepackage{lipsum}
\usepackage{natbib}

\input{../shorthandCommon}

%opening
\title{Vélabrögð}
\author{Helga Ingimundardóttir}

\begin{document}

\maketitle

%\begin{abstract}\end{abstract}

\section{Inngangur}
Í stuttu máli snýst verkefnið um að læra að bera kennsl á góðar lausnir. 
Ég hef einskorðað verkefnið við tilviksrannsókn á verkniðurröðun á vélum (JSP, 
e. job-shop scheduling problem) 
sem felst í því að þurfa að gera raðbundnar ákvarðanir um hvaða verk eigi 
að vera afgreitt næst, þar sem þau eru að keppast um sömu aðföngin.
Í raun má útvíkka aðferðafræðina til hvers kyns strjála bestun. 

Hugmyndina að rannsókninni kviknaði þegar ég var að vinna í raunhæfu verkefni í 
aðgerðagreiningu í grunnnámi mínu. Um var að ræða bestun á verkniðurröðun fyrir 
Össur. Í eðli sínu er verkniðurröðun einfalt verkefni, og skipar stóran sess 
hjá framleiðslufyrirtækjum en stærðargráðan á vandamálinu gerir það að verkum 
að oft er erfitt að leysa verkefnið með nákvæmum aðferðum. 
Þetta voru mín fyrstu kynni af því að þurfa að sætta mig við einhverja lausn 
sem var ekki endilega hin fræðilega ,,besta'' lausn. 
Hér koma brjóstvitsaðferðir (e. heuristics) eða ,,þumalputtareglur'' sterkar 
inn, en þá er stóra spurningin hvernig má koma á sjálfvirkni í því ferli?

\section{Verkniðurröðun á vélar}
Gerum ráð fyrir að við höfum $n\times m$ JSP, 
þar sem $n$ verk, $\mathcal{J}=\{J_j\}_{j=1}^n$, 
eiga að vera afgreidd á $m$ vélum, $\mathcal{M}=\{M_a\}_{a=1}^m$. 
Verkefnin þurfa að vera afgreidd í tiltekinni röð, þ.e. sérhvert verk $J_j$ 
þarf að fylgja runu af $m$ aðgerðum 
$\vsigma_j=\{\sigma_{j1},\sigma_{j2},\dotsc,\sigma_{jm}\}$. 
Rétt er að taka fram að verk getur ekki hafist handa á næstu vél fyrr en það 
hefur lokið núverandi aðgerð. 
Þar að auki getur sérhver vél aðeins meðhöndlað eitt verk í einu. 
Viðbættar skorður sem eru oft teknar til greina eru sleppitími og útgáfutími, 
en þeir eru ekki til skoðunar hér.
Markfallið er að tímasetja öll verk þannig að lágmarka skal hámarks heildartíma 
(e. makespan), $C_{\max}$. 

\begin{figure}\centering
    \includegraphics[width=0.8\textwidth]{figures/jssp_example.eps}
    \caption[Gantt rit af ókláraðri JSP stundaskrá]{Gantt rit af ókláraðri JSP 
    stundaskrá eftir 15 aðgerðir: heilir kassar tákna $\vchi$ og kassar með 
    brotalínu tákna $\mathcal{L}^{(16)}$. 
    Núverandi heildartími, $C_{\max}$, er gefinn upp sem punktalína.}
    \label{fig:jssp:example}
\end{figure}

Brjóstvitsaðferðir fyrir tímaáætlanir eru yfirleitt uppbyggingar- eða 
umbætunaralgrím.
Umbætunaralgrím byrja á fulbúinni lausn og reynir að finna sambærilegar, en 
betri lausnir. 
Uppbyggingaralgrím byrja með tómri lausn og bæta við einu verki í einu þar til 
lausnin er fullbúin, og er það sú nálgun sem aðferðafræðin mín gengur út frá. 
Í þessu tilfelli þá eru yfirleitt röðunarreglur (DR, e. dispatching 
rules) sem ákvarða hvaða ókláraða verk verður valið næst. Það er ekki nóg að 
vita hvaða verkefni ætti að vera valið næst, 
heldur þarf líka að athuga hvar væri best að staðsetja það. 
Þar sem við viljum búa til samþjappaðar tímaáætlanir  þá setjum við verkið af 
stað um leið og það er laust. 
Skoðnum nú Gantt ritið á mynd \ref{fig:jssp:example} sem sýnir dæmi um 
$6\times5$ JSP þar sem verknúmerið er gefið inn í kassa og er breidd hans 
vinnslutími verksins, 
vélarnar eru á lóðrétta ásinum og lárétti ásinn segir til um tíma 
og er núverandi $C_{\max}$ gefið sem punktalína. 
Búið er að setja af stað 15 aðgerðir, nefnilega, 
\begin{eqnarray}
\vchi=\left(J_3,J_3,J_3,J_3,J_4,J_4,J_5,J_1,J_1,J_2,J_4,J_6,J_4,J_5,J_3\right),
\end{eqnarray}
þar af leiðandi eru ókláruðu verkin eftirfarandi 
$\mathcal{L}=\{J_1,J_2,J_4,J_5,J_6\}$, sem lýsa þeim 5 mögulegu verkum sem geta 
verið afgreidd á tímapunkti $k=16$ (athugið að verk $J_3$ er fullklárað) -- 
þessi verk eru gefin upp með brotalínu og lýsa hvernig staðan gæti breyst. 
Við sjáum að $J_2$ getur verið staðsett á $M_3$ annaðhvort á milli  $J_3$ og 
$J_4$, eða eftir $J_4$.  Ef $J_6$ hefði nú þegar verið afgreitt, þá myndi 
myndast rauf á milli þess og $J_4$, þ.a.l. myndast þriðji möguleikinn, þ.e. 
fyrir $J_2$ er sett eftir $J_6$. 
Uppbyggingaralgrímið þarf því að ákveða hvert þessara raufa ætti að vera 
valið fyrir verkið og er það óháð röðunarreglunni sem er notuð. 
Mismunandi staðsetningaraðferðir geta verið skoðaðar, t.a.m. að velja þau rauf 
sem er minnsta (en nægjanlega stór) fyrir verkið. En grunnrannsóknir sýndu að 
slík nálgun gat í raun útilokað bestu lausn m.t.t. lágmarks heildartíma. 
En slík staða kom ekki upp ef við afgreiðum verkin um leið og þau berast. 

\section{Röðunarreglur}
Einfaldar röðunarreglur (SDR, e. single-based priority dispatching rule) 
(SDR), er fall af sérkennum verka/véla tímaáætlunarinnar. Sérkennin geta verið 
fastar eða breyst í takti við ferlið. Til dæmis getur forgangurinn byggst á 
eiginleikum vinnslutíma verkanna, 
\begin{description}
    \item[Minnsti vinnlutími (SPT, e. shortest immediate processing time)] 
    \hfill \\ gráðug aðferð sem klárar verk með minnsta vinnslutíma fyrst, 
    \item[Stærsti vinnslutími (LPT, e. longest immediate processing time)] 
    \hfill \\ gráðug aðferð sem klárar verk með stærsta vinnslutíma fyrst, 
    \item[Minnsta heildarvinna (LWR, e. least work remaining)] \hfill \\
    þar sem ásetningurinn er að klára verk sem eru komin langt á veg í 
    framvindu sinni, þ.e. að lágmarka verklistann $\mathcal{L}$,
    \item[Stærsta heildarvinna (MWR, e. most work remaining)] \hfill \\
    þar sem ásetningurinn er að flýta fyrir framvindu verka sem krefjast mikinn 
    vinnslutíma, og gefur því af sér jafnari framvidnu fyrir öll verk, aftur á 
    móti.
\end{description}
Þetta eru þær algengustu röðurnarreglur í starfi vegna einfaldleika þeirra og 
skilvirkni, en ótal fleiri reglur koma líka til greina. Yfirlit yfir 100 
sígildar röðunarreglur má finna í \citet{Panwalkar77}, en einnig er greinargóð 
lýsing á SDR eftir \citet{Haupt89}. 

\section{Tilraunir}
Við skulum skoða nú tvær gagnadreifingar fyrir JSP, nefnilega með slembinni 
uppröðun, kallað \texttt{j.rnd}, og einsleitri uppröðun, kallað \texttt{f.rnd}. 
Vinnslutíminn í báðum tilfellum er fengin með jafnri dreifinu á bilinu $[1,99]$.
Við höfum $N_{\textbf{train}}=500$ af hvorri dreifingu, og vitum bestu lausn. 
Þar sem vinnslutíminn er mismunandi þá munum við styðjast við að lágmarka 
frávik frá bestu lausn með eftirfarandi hætti, 
\begin{equation}
\rho=\frac{C_{\max}^{\text{DR}}-C_{\max}^{opt}}{C_{\max}^{\text{opt}}}\cdot 
100\%
\end{equation}
þar sem $C_{\max}^{\text{DR}}$ er fengið með DR og $C_{\max}^{\text{opt}}$ er 
besta lausn. 

Beitum nú þeim SDR sem við kynntum áðan, á gagnasettið okkar. Kassarit fyrir 
$\rho$ er gefið upp í mynd \ref{fig:boxplot}. Við sjáum að það er greinilegur 
munur á því hvaða SDR er beitt á gögnin, t.a.m. er MWR mjög hentug regla fyrir 
\texttt{j.rnd} aftur á móti er það ekki tilvikið fyrir \texttt{f.rnd} þar sem 
gagnstæð regla, LWR, kemur umtalsvert betur út. 

\begin{figure}\centering
    \includegraphics[width=0.8\textwidth]{figures/boxplot.pdf}
    \caption{Kassarit fyrir SDR}
    \label{fig:boxplot}
\end{figure}

Í mörgum tilfellum þá er látið staðar numið við þessar niðurstöður og sú DR sem 
kom best út er valin fyrir verkefnið. En það sem við viljum vita er hvað 
aðgreinir þessar reglur? Af hverju er svona mikill munur á niðurstöðum? 
Sérstaklega í ljósi þess að innblástur þeirra virðist ekki vera svo ólíkur, SPT 
svipar til LWR og LPT til MWR. Einnig er SPT andstæða LPT og LWR fyrir MWR. Af 
hverju er einsleit verkumröðun \texttt{f.rnd} að velja LWR fram yfir MWR og 
öfugt? Einnig hvenær fer að greina á gæði lausnanna með þessum aðferðum?

Ef við skoðum bestu lausnir fyrir verkefnin og athuga hvenær þær væru 
jafngildar að beita SDR, líkt og mynd \ref{fig:diff:opt:SDR} sýnir, þá fyrir 
\texttt{j.rnd} má sjá að oftar en ekki er LWR og LPT verri en að velja reglu að 
handahófi (brotalína) sem útskýrir slakar niðurstöður þeirra. 
Einnig má sjá að SPT sýnir hegðun sem er líklegri til að 
hagstæðri en MWR -- en aðeins til að byrja með. Eftir það fer MWR að vera 
líklegri til vinnings. En því miður er ekki hægt að beita SPT fyrst (segjum frá 
skrefi 1-10) og láta svo MWR taka svo við, því þá erum við nú þegar búin að 
stilla tímaáætluna á þann veg að MWR nær ekki að rétta sig af. 

\begin{figure}\centering
    \includegraphics[width=0.8\textwidth]{figures/{stepwise.6x5.OPT.SDR}.pdf}
    \caption{Líkurnar að SDR sé hagstæðust, brotalínan lýsir að verk að 
    handhófi sé hagstæðast.}
    \label{fig:diff:opt:SDR}
\end{figure}

Aftur á móti má sjá, hvernig eru reglurnar að breytast frá öðrum verkum sem 
fall af tíma. Mynd \ref{diff:case:track:6x5} sýnir hvernig frávikið $\rho$ er 
að breytast sem fall af ákvörðunarskrefi með því að fylgja ákveðinni SDR 
(brotalína). Einnig er skoðað hvernig væru besta og versta frávik frá þeirri 
sömu stöðu, þ.e. ef við fylgdum ekki reglunni. 
Þá sést glögglega að MWR er stöðugri í að koma sér í betri stöðu en SPT, 
jafnvel þótt bestu lausnirnar virtust fylgja SPT hegðun til að byrja með. 

\begin{figure}\centering
    \includegraphics[width=0.8\textwidth]{figures/{stepwise.6x5.Track.casescenario}.pdf}
    \caption{Breyting frávik sem fall af áætlunarskrefi fyrir gefið SDR.}
    \label{diff:case:track:6x5}
\end{figure}



%          Problem Dimension    Q1    Q3
%          j.rnd.6x5   j.rnd       6x5 19.91 47.21
%          f.rnd.6x5   f.rnd       6x5 18.46 35.52


\bibliographystyle{unsrtnat}
\bibliography{../references}  
\end{document}
